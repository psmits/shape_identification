\documentclass{letter}
\usepackage{microtype}
%\usepackage{letterbib}
%\usepackage{natbib}
\frenchspacing

\signature{Peter D Smits}
\address{Committee on Evolutionary Biology \\ University of Chicago \\
1025 E. 57th Street \\ Culver Hall 402 \\ Chicago, IL 60637 \\
psmits@uchicago.edu}

\begin{document}
\begin{letter}{Editor \\ \textit{Systematic Biology}}
  \opening{Dear Editor,}

  Please find enclosed the manuscript entitled ``How cryptic is cryptic diversity? Machine learning approaches to classifying morphological variation in the Pacific Pond Turtle (\textit{Emys marmorata})'' which we are submitting for consideration at the journal \textit{Systematic Biology}. This manuscript is authored by myself (Peter D Smits), Kenneth D Angielczyk, and James F Parham. We believe our analysis and results are of interest to the community as we present a non-standard approach for evaluating the intensity of cryptic diversity and the efficacy of different classification schemes.

  Morphological and molecular attempts to classify the turtle species \textit{Emys marmorata} have produced three different potential sub-species or species divisions. In this study we compared these hypotheses of how to subdivide \textit{Emys marmorata} using multiple machine learning approaches. We analyzed over 500 specimens of this species from over its entire geographic range. Additionally, instead of classification accuracy, we use a metric of the ratio of true-positive and false-positive rates for both in-sample and out-of-sample model performance. Finally, in order to test whether our approaches were valid, we compared these results to those from the analysis of seven morphologically distinct turtle species.

  Our results demonstrate, that while our approaches can very precisely identify morphologically distinct groupings, the morphological diversity of the \textit{Emy marmorata} species ``complex'' is inconsistent with any of the proposed classification schemes. We propose that these results may be due to the possibility that genetic differentiation is not associated with plastron shape variation \textit{below} the species level and/or that local selective pressures (e.g., from hydrological regime) overwhelm morphological differentiation. Additionally, a reconsideration of the methods used to delimit \textit{E. ``pallida,''} the lack of barriers to gene flow, the strong evidence for widespread admixture between lineages, and the fact that plastron shape can be used to differentiate other emydine species suggest that its lack of diagnosability most likely reflects the non-distinctiveness of this proposed taxon.
  
  Thank you for considering our work. Please send all correspondence regarding this manuscript to me via my email address (psmits@uchicago.edu).

  \closing{Sincerely,}

  \encl{Article; supplementary text, figures, tables.}

\end{letter}
\end{document}
